\documentclass[twocolumn,twoside,a4paper,10pt]{IEEEtran}

\usepackage{bm}
\usepackage[utf8]{inputenc}
\usepackage[T1]{fontenc}
\usepackage[noadjust]{cite}
    \renewcommand{\citepunct}{,\penalty\citepunctpenalty\,}
    \renewcommand{\citedash}{--}
\usepackage{lipsum}
\usepackage{url}
\usepackage{graphicx}
% ADD THE PACKAGES YOU MIGHT NEED
\usepackage{amsmath}
\usepackage{amssymb}
\usepackage{amsthm}
\usepackage[future]{hyperref}
\usepackage{cleveref}
\usepackage{xcolor}
\usepackage[ruled]{algorithm2e}
\SetKwComment{Comment}{/* }{ */}


\newtheorem{definition}{Definition}
\newtheorem{proposition}{Proposition}
\newtheorem{theorem}{Theorem}
\newtheorem{lemma}{Lemma}


% % % % % % % % % % % % % % % % % %
%     EDIT THE THESIS' DETAILS    %
% % % % % % % % % % % % % % % % % %


\usepackage[english]{babel}     % Use either `english`, `catalan` or `spanish` to change titles and other template text
\title{Primal-dual Unfolding for\\ Underwater Digital Images}
\author{Frank William Hammond Espinosa}
\email{frank.william.hammond@gmail.com}
\tutors{Julia Navarro Oliver and Ana Belén Petro Balaguer}
\specialization{Artificial Intelligence}
\academicyear{2024/25}
\keywords{Underwater imaging, Applied mathematics}
% \showedisslogo  % Uncomment this command if you are an EDISS student.
\newcommand{\Frank}[1]{\textcolor{red}{#1}}
\DeclareMathOperator*{\argmin}{\operatorname*{argmin}}
\DeclareMathOperator*{\argmax}{\operatorname*{argmax}}

% \setlength{\parindent}{0px}

\crefname{algorithm}{algorithm}{algorithms}
\Crefname{algorithm}{Algorithm}{Algorithms}

\DeclareMathOperator*{\argmax}{arg\,max}
\DeclareMathOperator*{\argmin}{arg\,min}

\begin{document}

\thispagestyle{empty}

\makeatletter
\begin{titlepage}

    %%%%
    % COBERTA
    %%%%
    
    \bfseries 

    {
    \includegraphics[height=30mm]{figures/logos/UIB_UniversitasBalearica_horizontal.png}
    \ifedisslogo
        \hfill
        \raisebox{-10mm}{
            \includegraphics[height=40mm]{figures/logos/EDISS.jpeg}
        }
    \fi
    }
	
    \vspace{20mm}
	{
        \LARGE 
        
	    \iflanguage{spanish}{TRABAJO DE FIN DE MÁSTER}{%
        \iflanguage{catalan}{TREBALL DE FI DE MÀSTER}{%
                             MASTER'S THESIS%
        }}
        \vspace{20mm}
        
        \hfill%
        \begin{minipage}{130mm}
            \begin{flushright}\scshape \bfseries \@title\end{flushright}
        \end{minipage}
        \vspace{20mm}
        
	    \@author
        \vspace{20mm}
    }
    {
        \large
        
	    \iflanguage{spanish}{Máster Universitario en Sistemas Inteligentes (MUSI)}{%
        \iflanguage{catalan}{Màster Universitari en Sistemes Intel·ligents (MUSI)}{%
                             Master's Degree in Intelligent Systems (MUSI)}}
        \vspace{0mm}
        
	    \iflanguage{spanish}{Especialización en \@specialization.}{%
        \iflanguage{catalan}{Especialització en \@specialization.}{%
                             Specialization in \@specialization.}}
        \vspace{5mm}

	    \iflanguage{spanish}{Centro de Estudios de Postgrado}{%
        \iflanguage{catalan}{Centre d'Estudis de Postgrau}{%
                             Centre for Postgraduate Studies}}
        \vspace{15mm}

	    \iflanguage{spanish}{Año académico {\@academicyear}}{%
        \iflanguage{catalan}{Any acadèmic {\@academicyear}}{%
                             Academic year \@academicyear}}
    }
    \thispagestyle{empty}
    \newpage

    %%%%
    % PORTADA
    %%%%
    
	
    \vspace*{20mm}
	{
        \Huge 
        \bfseries 
        
        {\scshape \@title}
        \vspace{10mm}
        
        \@author
        \vspace{20mm}
    }
    {
        \LARGE
        \bfseries
        
	    \iflanguage{spanish}{TRABAJO DE FIN DE MÁSTER}{%
        \iflanguage{catalan}{TREBALL DE FI DE MÀSTER}{%
                             MASTER'S THESIS}}
        \vspace{5mm}

	    \iflanguage{spanish}{Centro de Estudios de Postgrado}{%
        \iflanguage{catalan}{Centre d'Estudis de Postgrau}{%
                             Centre for Postgraduate Studies}}
        \vspace{5mm}

	    \iflanguage{spanish}{Universidad de las Islas Baleares}{%
        \iflanguage{catalan}{Universitat de les Illes Balears}{%
                             University of the Balearic Islands}}
        \vspace{10mm}   
    }
    {
        \bfseries
   
	    \iflanguage{spanish}{Año académico {\@academicyear}}{%
        \iflanguage{catalan}{Any acadèmic {\@academicyear}}{%
                             Academic year \@academicyear}} 
        \vspace{20mm}
        
    }
    {
	    \iflanguage{spanish}{Palabras clave:}{%
        \iflanguage{catalan}{Paraules clau:}{%
                             Keywords:}}
        \vspace{5mm}   
        
        \@keywords
        \vspace{30mm}
    }
    {
        \itshape

        Tutor(s): \@tutors.
    }
\end{titlepage}
\makeatother

\maketitle


% % % % % % % % % % % % % % % %
%     EDIT THE MANUSCRIPT     %
% % % % % % % % % % % % % % % %

\begin{abstract}
\noindent Lorem ipsum etc.
\end{abstract}
\section{Introduction and related works}

Underwater image processing is of great interest for its applications on marine robotics \cite{956031} or marine biology \cite{4099139}. It presents a unique set of challenges due to two
physical phenomena that govern the image formation process in aquatic media. As light propagates through the water, it loses its intensity in a process called \textit{absorption}, at a significant rate that is dependent both on the condition of the water and the light
wavelength. Specifically, longer wavelengths tend to be absorbed more rapidly, resulting in underwater images often appearing bluish or greenish to the human eye. In addition to losing intensity, light can change directions when colliding with
suspended particles in the medium. This process, known as \textit{scattering}, results in a loss of contrast and hazy appearance. Inverting these two processes
allows the recovery of a much richer image that is better suited for subsequent imaging tasks such as annotation \cite{lisani2022analysis} or classification \cite{1707999}. Such inversion is particularly challenging and therefore traditional, general approaches like
histogram equalization or Retinex models are insufficient without modifications
specific to the underwater environment \cite{xie2021variational}.

McGlamery \cite{10.1117/12.958279} and Jaffe \cite{50695} introduced what is now the most widely adopted mathematical framework for this problem, now known as the
Undewater Image Formation Model (UIFM). Modern approaches often rephrase a simplified version of the original models as \textit{dehazing} problems \cite{GALDRAN2015132}. This simplified version of the UIFM does not take into account its \textit{forward scattering} component, but still produces acceptable results. Most importantly, it yields equations very similar to those of the Atmospheric Scattering Model (ASM), which is mostly associated to dehazing problems, and therefore many techniques
used for atmospheric dehazing can be adapted for the aquatic environment.

In order to solve these dehazing problems, many approaches first try to estimate the \textit{background light} (a.k.a. \textit{atmospheric light} in dehazing settings) of the scene via some
heuristic, which often uses some sort of statistical measure of the intensities
in one or several color channels of the image. A simple approach consists of using the brightest pixel in the image, but due to its brittleness it has been superseded by more sophisticated techniques. Popular approaches are based
on the dark channel intensity histogram \cite{5567108}, a quad-tree algorithm that searches the brightest areas in
the image \cite{xie2021variational,kim2013optimized,li2016underwater}, on the brightest local intensity minimum \cite{6104148}, or modifications of the above and combinations thereof.

Once the background light has been approximated, an array of techniques are often used to estimate the \textit{transmission map}. Many traditional approaches (\cite{xie2021variational,GALDRAN2015132,Drews_2013_ICCV_Workshops}) are based on the classical Dark Channel Prior \cite{5567108} or versions of it, such as the Red Channel Prior \cite{GALDRAN2015132} or the Underwater Dark Channel Prior \cite{Drews_2013_ICCV_Workshops}. Other
techniques are possible too, such as the one presented in \cite{7574330}, which
minimizes information loss due to RGB encoding. Having estimated the background light and transmission map, it is then possible to analytically invert the simplified UIFM
and obtain an estimate for the scene radiance, which effectively corresponds to
the undegraded image.

Machine Learning (ML) approaches have been successfully used for these tasks too. A classification of different techniques used for dehazing is presented in \cite{10.1145/3576918}. Among the wide variety of classes collected, two follow the paradigm described above: those that learn the transmission map only, and those that jointly learn the transmission map and the background light.

In \cite{7539399}, authors estimate the background light by selecting the brightest pixel below a certain threshold, and then feed the unaltered input into a small neural network they coined \textit{DehazeNet}, which then outputs a coarse transmission map. After refining this map with a guided filter, they invert the ASM to obtain the scene radiance. A similar technique
is used in \cite{8450630}.
The algorithm in \cite{9190777} is also alike, but instead of yielding the tranmission map directly, they use the output
of the penultimate layer of their nework as the transmission map, and then pass
the resulting radiance estimation through a final, linear layer. A ``neural-network-first'' approach that is still based on the ASM can be found in 
\cite{guo2019dense}. Authors propose two similar encoder-decoder architectures in which a single encoder feeds either two or three decoders with different purposes: one estimates the atmospheric light, another the transmission map and,
when present, the third approximates the scene radiance directly. In both configurations,
the final output is computed via the ASM solely based on the atmospheric light
and transmission map estimates. However, in the three-decoder architecture, the
output of the third decoder is used as assistance during training, since, according to the authors, this
improves performance for severely degraded images.

Not all methods are based directly on inverting a physical model. These are sometimes called \textit{model-free} algorithms. Examples of this are fusion-based algorithms \cite{6247661,ancuti2017color}, which compute several auxiliary
images from the input - each intended to enhance a certain aspect of it - and then combine those to form a definitive output. The way of computing these auxiliary images often does incorporate aspects of the UIFM. Most ML-based
models presented in \cite{10.1145/3576918} that do not fall under the two previously mentioned categories can also be considered \textit{model-free}. Some draw inspiration from either the ASM \cite{Li_2017_ICCV} or 
from Retinex theory \cite{9274531} to justify their models' architectures, but 
ultimately do not leverage a physical model that explains image formation. 
However, their lack
of interpretability is often somewhat compensated by exhibiting excellent performance.

Algorithm \textit{unfolding}, also called algorithm \textit{unrolling}, presents a way of leveraging physical models while
using neural networks in a way that allows them to be trained end-to-end,
yielding fully explainable models that can still achieve competitive results.
An excellent exposition of the core ideas about unfolding and a collection of
illustrative approaches is presented in \cite{9363511}. Briefly, it consists in
first solving an optimization problem via an iterative approach,
and then replacing some part of the iterative scheme by a neural network. The seminal work that introduced the technique \cite{10.5555/3104322.3104374} illustrates it quite well. It was performed in the context of sparse coding, which can be translated into minimizing the functional
\[
  \frac12\|X-W_dZ\|_2^2 + \alpha \|Z\|_{1}
\]
with respect to \(Z\in \mathbb{R}^m\) for a given \(X\in \mathbb{R}^n\) and a maximally ranked matrix of choice \(W_d\in \mathcal{M}_{n\times m}(\mathbb{R})\), usually with \(m\gg n\). A popular
iterative scheme for solving this problem, the Iterative Shrinking and Tresholding Algorithm (ISTA), consists of the following iterates:
\[
 z_0 = 0, ~~~~ z_{n+1} = S_\lambda\left(W_tz_n+W_eX\right)
,\]
where \(W_t\) and \(W_e\) are fixed matrices that can be analytically computed
from \(W_d\), and \(S_\lambda\) is a function also with an analytically closed
form that depends only on the parameter \(\lambda>0\) (said closed forms are not included here for brevity). One of the main contributions of \cite{10.5555/3104322.3104374} was the Learned ISTA. In that algorithm, the parameters \(\lambda\), \(W_t\) and \(W_e\) are the output of a neural network that takes as inputs different values of \(X\), and the
iterative scheme above is repeated for a fixed number of iterations. During training,
the output is evaluated with the target functional for a pre-fixed matrix \(W_d\), enabling end-to-end training
of the network for optimizing the parameters. This resulted in a fully
explainable model that achieved impressive performance at the time, which in
this case meant faster convergence to an optimum value of \(Z\).

Since then, many similar approaches have been developed in a variety of fields.
Iterative schemes containing some sort of proximity operator are a natural
fit for unfolding, since the proximity operator itself can be replaced by a neural
network in-place. This is particularly useful when the proximity operator
derives from a regularization function, since those are often chosen somewhat
arbitrarily in inverse imaging problems, and learning the regularization prior
from actual data is both practical and mathematically elegant. The Chambolle-Pock algorithm, also known as primal-dual hybrid gradient algorithm, offers a uniquely
direct way of achieving this structure, so much so that it is possible to
perform this substitution in a fully general manner, as authors in \cite{8271999} suggest.

In this work, heuristics from the traditional approaches outlined above are used to estimate the background light and transmission map - concretely, the quad-tree and RCP methodologies. Then, a variational problem is posed for the
full UIFM, as opposed to most traditional approaches which use only the simplified version. The problem is theoretically studied, ensuring partial well-posedness for a wide family of regularizations, and an iterative scheme is
drawn with the Chambolle-Pock algorithm. Finally, said scheme is unfolded by
substituting part of the regularization by a neural network, which is trained
end-to-end with the UIEB dataset. Two neural network architectures are used
to this end: a classical UNet, and a novel one baptized as AUNet (Additive UNet),
which is a simple modification to the UNet that changes concatenation on the
skip connections by additions.
\section{Mathematical model}

In this section, all of the mathematical machinery and notation needed to understand and formalize the problem is presented (\cref{subsec:theoretical-grounding}), the physical equations are introduced and manipulated for convenience (\cref{subsec:physical-model}), and the problem is formalized and theoretically studied (\cref{subsec:formalization}).

\subsection{Theoretical grounding and notation}\label{subsec:theoretical-grounding}
Although the mathematical exposition in this section is, for the most part, either self-contained or specifically referenced, a certain degree of familiarity with modern mathematical analysis is assumed. A minimum familiarity with digital imaging is also assumed.

Although not recommended, the reader that is either already very familiar with higher mathematical analysis and convex optimization, or too unfamiliar with it, may skip this section and read \cref{subsec:physical-model} and below.

\subsubsection{Image representations}

Two representations of digital images will be used in this work, the equivalences between which will be described below.

The most common representation among computing frameworks is that of multidimensional arrays, sometimes also called \textit{tensors}. We shall refer to this one as the \textbf{discrete} representation. Since \textit{pytorch} will be used to implement the algorithms described in this text, the standard 4D representation will be used; that is, images will always have the \textit{Shape} \((B, C, H, W)\), where \(B\) is the batch size (\(1\) for a single image), \(C\) is the number of channels (\(3\) for an RGB image, \(1\) for a grayscale one), and \(H\) and \(W\) are the height and width of the image, respectively.

The other representation used will be called \textbf{continuous}\footnote{The term \textit{continuous} here refers to the domain of the representation. In fact, the functions involved will almost never be continuous.}. It will be the preferred one in the mathematical formulae of this text. The idea is to think of images as either scalar or three-dimensional vector fields for grayscale or RGB images, respectively, defined on a rectangular domain \(\Omega\subseteq \mathbb{R}^2\). Said fields are always piecewise constant on right-semiclosed unit squares with sides parallel to the coordinate axes.

In order to be able to comfortably extend continuous operations for images, it is useful to center the domain \(\Omega\) with respect to the origin. Concretely, for a batched image \(I\) with shape \((B, C, H, W)\), \(B\)  fields are defined on the same domain \(\Omega\). Namely, \(I^1, \dots, I^B\colon\Omega\to \mathbb{R}^C\), where the domain \(\Omega\) is given by
\[
  \Omega = (-\lceil \tilde{w}\rceil - 1, \lfloor \tilde{w}\rfloor]\times(-\lceil \tilde{h}\rceil -1, \lfloor \tilde{h}\rfloor]
,\]

with \(\overline{w}=\frac{W-1}{2}\) and \(\overline{h}=\frac{H-1}{2}\). The relationship between the two representations is given by

\begin{equation}\label{eq:discrete-continuous}
  I[b][c][h][w] = I^{b+1}_{c+1}(w - \lceil\tilde{w}\rceil, h - \lceil\tilde{h}\rceil)
\end{equation}

where the left term follows a standard \(0\)-indexed array notation, and the right is standard mathematical function notation. Then, the functions \(I^b_c\) are extended to the rest of \(\Omega\) by imposing that they be constant on all right-semiclosed squares \((x-1,x]\times(y-1, y]\).

It is also possible to extend the domain of the continuous representations to all of \(\mathbb{R}^2\) by simply setting them to \(0\) outside of \(\Omega\). Reciprocally, for a given set of functions \(I^1, \dots, I^B\colon \mathbb{R}^2\to \mathbb{R}^C\), it is possible to obtain a discrete representation of a digital image by fixing the desired dimensions \(H\) and \(W\) and sampling as in \cref{eq:discrete-continuous}.

It is immediate to see that this identification between the two representations respects pointwise operations (in particular, function sums and products). It is also possible to see that, for an appropriate translation, the convolution defined in \textit{pytorch} with a kernel \(g\) coincides with the continuous convolution with (the continuous version of) the inverted kernel \(\overline{g}(x) = g(-x)\).

\subsubsection{\(L^p\) spaces}
It will be useful to utilize mathematical terminology and machinery that is best exposed in function spaces. The most important and suitable function spaces that will be employed are the \(L^p\) spaces.

\begin{definition}
  Let \(\Omega\) be a Borel measurable subset of \(\mathbb{R}^n\) for some \(n\in \mathbb{Z}^+\), and \(\mu\) the Lebesgue measure on the Borel sets of \(\mathbb{R}^n\).\footnote{Properly defining the Borel sets and the Lebesgue measure is out of the scope of this work. All rectangles, either open, closed or semiclosed are Borel measurable subsets of \(\mathbb{R}^2\). All integrals with respect to the Lebesgue measure coincide with their classical Riemann integral counterpart, whenever the latter is defined.} For a given measurable function \(f\colon \Omega\to \overline{\mathbb{R}}\), where \(\overline{\mathbb{R}}\) is the extended real line equipped with its usual topology, define the quantity

  \[
    \|f\|_{p} = \left(\int_{\Omega}|f|^p~d\mu\right)^{\frac 1p}
  \]

  for each \(p\in[1, +\infty)\). Define the set of \(p\)-integrable functions as
  \[
    \mathcal{L}^p(\Omega) = \left\{f\colon\Omega\to \mathbb{\overline{R}}~\left|~\|f\|_{p}<+\infty\right.\right\}
  .\]

  It can be shown that \(\mathcal{L}^p(\Omega)\) is a normed vector space when equipped with \(\|\cdot\|_{p}\) as a norm.

  Finally, define \(\bm{L^p(\Omega)}\) by identifying functions in \(\mathcal{L}^p(\Omega)\) that coincide almost everywhere with respect to \(\mu\) and equipping the resulting equivalence classes with the norm of any of their elements. It is possible to show that \(L^p(\Omega)\) is a separable Banach space for any \(p\), and that it is a Hilbert space for \(p=2\).
\end{definition}

It is trivial to see that the continuous representation of a digital image \(I\) with domain \(\Omega\) is always an element of \(L^p(\Omega)\) for any \(p\in[1, +\infty)\).

\subsubsection{Functional and convex analysis} It is possible to develop in a surprising generality a number of optimization algorithms which can later be applied for image processing problems. The continuous representation of digital images described above will allow the work with operators defined on function spaces in a more abstract fashion than with multidimensionals arrays. By doing so, the involved operators and formulae obtained often have much cleaner closed forms.

In this section, a few relevant definitions are presented. Most importantly, the direct method is introduced and proved. This theorem will be employed later to show the well-posedness of the presented minimization problems. Some results are given without proof; the interested reader can consult chapters 1 to 3 of \cite{clason2024introductionnonsmoothanalysisoptimization}.

\begin{definition}
  A normed vector space \(X\) is said to be a \textbf{Banach space} if it is complete under the metric induced by its norm. A \textbf{functional} on \(X\) is a function \(F\colon X\to\mathbb{R}\cup\{+\infty\}\).
\end{definition}

\begin{definition}
  The \textbf{dual space} of \(X\) is defined as the set of all continuous linear functionals on \(X\) with codomain on \(\mathbb{R}\), where continuity is to be interpreted with respect to the topology induced on \(X\) by its norm and the usual topology in \(\mathbb{R}\). We denote it as \(X^*\). It can be shown that \(X^*\) is a Banach space when equipped with the operator norm
  \[
    \|x^*\|_{X^*} = \sup_{x\in X, \|x\|=1}\|x^*(x)\|
  .\]

  We say that a given sequence \(\{x_n\}_{n\in\mathbb{N}}\subseteq X\) \textbf{converges weakly} to a given \(x\in X\) if the sequence \(\{f(x_n)\}_{n\in\mathbb{R}}\) converges (in \(\mathbb{R}\)) to \(f(x)\) for every \(f\in X^*\).
\end{definition}

It is immediate to check that the usual convergence of a sequence in \(X\) implies its weak convergence to the same limit.

\begin{definition}
  A given functional \(F\) on a Banach space \(X\) is said to be \textbf{weakly lower semicontinuous} if, for each weakly converging sequence \(\{x_n\}\subseteq X\) with weak limit \(\lim_nx_n\in X\),

  \[
    F(\lim_nx_n)\leq \liminf_n F(x_n)
  .\]

  Importantly, the norm of \(X\) is a weakly lower semicontinuous functional.
\end{definition}

\begin{definition}\label{def:convexity}

  A given functional \(F\) on \(X\) is said to be \textbf{convex} whenever the condition
  \[
    F(\lamdba x + (1-\lambda)y) \leq \lambda F(x) + (1-\lambda) F(y)
  \]
  holds for any \(x, y\in X\) and any \(\lambda\in(0, 1)\). If, additionally, the inequality is strict whenever \(x\neq y\), \(F\) is said to be \textbf{strictly convex}.
\end{definition}

\begin{definition}
  A functional \(F\) is said to be \textbf{coercive} if, for every sequence \(\{x_n\}\subseteq X\) such that \(\lim_n \|x_n\|=+\infty\), then \(\lim_nF(x_n)=+\infty\).
It is said to be \textbf{proper} if there exists some \(x\in X\) such that \(F(x) < +\infty\).
\end{definition}

\begin{definition}
  Let \(X\) be a Banach space. The \textbf{bidual} space of \(X\), denoted by \(X^{**}\), is the dual space of its dual space, that is, \(X^{**}=(X^*)^*\). There is a canonical linear mapping from \(X\) to \(X^{**}\) given by
  \[
    x\mapsto(x^*\mapsto x^*(x))
  \]
  which can be shown to always be isometric and therefore injective. \(X\) is said to be \textbf{reflexive} whenever the canonical mapping above is also surjective.
\end{definition}

It is a direct consequence of the Banach-Alaoglu theorem (which is omitted here for brevity) that every bounded sequence in a reflexive Banach space has a weakly converging subsequence.

\begin{theorem}[The direct method]\label{thm:direct-method}
  Let \(X\) be a reflexive Banach space and let \(F\) be a proper, coercive and weakly lower semicontinuous functional on \(X\). Then, the minimization problem
  \[
    \min_{x\in X} F(x)
  \]
  admits a solution. Moreover, if \(F\) is strictly convex, the solution is unique.
\end{theorem}
\textit{Proof.} For the existence, note that since \(F\) is proper, \(\inf_{x\in X} F(x)<+\infty\). It is therefore possible to construct a sequence \(\{x_n\}_n\subseteq X\) such that \(F(x_n)<+\infty\) for all \(n\) and \(F(x_n)\to \inf_{x\in X}F(x)\). Since \(\{F(x_n)\}_n\) is a converging sequence of real numbers, it is bounded. This, together with the coerciveness of \(F\), implies that \(\{x_n\}_n\) is bounded in norm. Since \(X\) is reflexive, it has a weakly converging subsequence. Denote the subsequence by \(\{\overline{x}_n\}_n\) and its limit by \(\overline{x}\in X\). Then, utilizing the weak lower semicontinuity of \(F\):
\[
  \inf_{x\in X}F(x) \leq F(\overline{x})=F(\lim_n \overline{x}_n)\leq\liminf_nF(\overline{x}_n)=\inf_{x\in X}F(x)
.\]
This shows that \(\overline{x}\) is a minimizer of \(F\). The uniqueness for the strict convexity is immediate by contradiction. \(\qed\)

Finally, a series of ``bespoke'' lemmas with simple proofs will be useful further ahead.

\begin{lemma}\label{lemma:sum-of-lsc}
  The sum of two weakly lower semicontinuous functionals is weakly lower semicontinuous.
\end{lemma}

\textit{Proof.} Immediate using the definition and the fact that \(\liminf_n{a_n} + \liminf_n{b_n}\leq \liminf_n{a_n + b_n}\). \(\qed\)

\begin{lemma}\label{lemma:precomposition-with-linear}
  Let \(X\) and \(Y\) be Banach spaces. Let \(F\colon Y\to \mathbb{R}\cup\{+\infty\}\) be a weakly lower semicontinuous functional, and \(K\colon X\to Y\) a continuous function. Then, \(F\circ K\) is weakly lower semicontinous.
\end{lemma}

\textit{Proof.} Since \(K\) is continuous, \(K(\lim_n x_n) = \lim_n K(x_n)\) for any converging sequence \(\{x_n\}_n\subseteq X\). The result is immediate by applying the definition of weak lower semicontinuity. \(\qed\)

\subsubsection{Convex optimization} Although necessary to establish the existence of solution, the direct method offers no way to actually compute the minimum of a posed problem. Other techniques, which often involve iterative algorithms, are useful for this. Notably, the Chambolle-Pock primal-dual algorithm will be used in this work. The rest of this section is devoted to briefly present the concepts necessary to enunciate it. For a more detailed explanation, the reader is referred to chapters 3 to 7 and section 8.4 of \cite{clason2024introductionnonsmoothanalysisoptimization}.

\begin{definition}
  Let \(X\) be a Banach space and \(F\) be a proper functional on \(X\). Define \(F^*\colon X^*\to\mathbb{R}\cup\{+\infty\}\) by
  \[
    F^*(x^*) = \sup_{x\in X}x^*(x) - F(x)
  .\]
This function is called the \textbf{Fenchel conjugate} or \textbf{convex conjugate} of \(F\).

  If \(Y\) is another Banach space, and \(A\colon X\to Y\) is a continuous linear operator, then a linear operator \(A^*\colon Y^*\to X^*\) is defined by
  \[
     (A^*y^*)(x) = y^*(Ax)
  .\]
This is called the \textbf{conjugate} operator of \(A\)\footnote{Although the notation and name used are the same, these conjugates are essentially different concepts. In this text there will be no ambiguity with these notations because the codomain of a linear operator whose conjugate is needed will never be the real or extended real numbers. Therefore, there is no ambiguity in always considering the conjugate of a given operator as the linear conjugate when the operator is linear and the convex conjugate otherwise.}.
\end{definition}

\begin{definition}
  Let \(X\) be a Banach space and \(F\) be a convex, weakly lower semicontinuous functional on \(X\) that is either bounded below or proper. For a given \(\tau>0\), define the \textbf{proximity operator} of \(F\) with step size \(\tau\) as
  \[
    \text{\emph{prox}}_{\tau F}(x) = \argmin_{y\in X}F(y) + \frac{\|x-y\|^2}{2\tau}
  .\]
This operator is well defined by \cref{thm:direct-method}\footnote{It is possible to define this operator in greater generality, but doing so involves theory about subdifferentials and monotone operators that is out of the scope of this work.}.
\end{definition}

A remark to be made here is that, for separable Hilbert spaces (Banach spaces in which the norm is induced by an inner product), the dual space \(X^*\) is,
in a very strong sense, completely equivalent to the space itself, \(X\). Namely,
the mapping \( x\mapsto (y\mapsto \langle x,y\rangle)\) (where \(\langle\cdot,\cdot\rangle\) is the inner product in \(X\)) defines an isometric isomorphism between \(X\) and \(X^*\)\footnote{The surjectivity of this mapping is sometimes called the \textit{Riesz-Fréchet} representation theorem.}. For this reason, dual spaces will be tacitly assumed to the base spaces in the following
theorem.

\begin{theorem}[Chambolle-Pock algorithm]\label{thm:chambolle-pock}
  Let \(X\) and \(Y\) be separable Hilbert spaces. Let \(F\colon X\to\overline{\mathbb{R}}\), \(G\colon Y\to\overline{\mathbb{R}}\) be proper, convex and lower semicontinuous functionals, and let \(A\colon X\to Y\) be linear and continuous. If the problem
  \[
    \min_{x\in X}F(x) + G(Ax)
  \]
  has a solution, then the sequence \(\{x_n\}_n\subseteq X\) defined by the following iterates converges to it:
  \begin{equation}\label{eq:iterates}
    \left\{
    \begin{split}
      x_{n+1} & = \text{prox}_{\tau F}(x_n - \tau A^*z_n) \\
      \tilde{x}_{n+1} & = 2x_{n+1} - x_n \\
      z_{n+1} & = \text{prox}_{\sigma G^*}(z_n + \sigma A\tilde{x}_{n+1})
    \end{split}
    \right.
  \end{equation}
\end{theorem}

\subsection{Physical model and formalization of the problem} \label{subsec:physical-model}
\subsubsection{Underwater Image Formation Model}
When light travels in a water medium, two relevant physical phenomena different from those found in an aerial medium appear: first, the intensity in this light
is absorbed by the water molecules in a significant manner. Such absorption is
also dependent on the wavelength of the light, and is perceived by the human vision system (HVS) as the red color being underrepresented. The second phenomenon is
\textit{scattering}, whereby light does not travel in a straight line from the
object to the camera forming an infinitely thin beam, but it is reflected on water molecules instead, forming a cone. This is perceived by the HVS as the image
presenting a glow-like or hazy appearance.

Classical works in underwater imaging by McGlamery \cite{10.1117/12.958279} and Jaffe \cite{50695} effectively quantify these phenomena and model the formation of an underwater image \(I\) as a linear superposition of three components, in what is often referred to as the Underwater Image Formation Model (UIFM) \cite{xie2021variational}:
\[
  I = I_{d} + I_{fs} + I_{bs}
.\]
The \(I_d\) (\textit{direct}) term corresponds to the light that travels in a straight line from the object to the camera; the \(I_{fs}\) (\textit{forward scattering}) component corresponds to light that travels from the object to the camera but is scattered in a small angle; and the \(I_{bs}\) (\textit{backward scattering}) term corresponds to light that travels directly from the light source to the camera by being scattered in a large angle.

Importantly, the forward-scattering component can be obtained from the direct component via a convolution with a \textit{point-spread function} \(g\).

These classical models are based purely on physical properties and depend heavily on the
conditions of the water and camera, presenting several constants that need to be
estimated experimentally. This poses a disadvantage when working with
underwater images taken in unknown conditions, as is often the case in
general-purpose underwater image processing algorithms.

For this reason, some modifications and simplifications of the original models
have been proposed over the years. Notably, similar notation to the following is used commonly in modern literature \cite{xie2021variational,ancuti2017color}: let \(I\) be the captured image (i.e., the input), \(J\) the scene radiance (i.e., the desired image), \(t\) the transmission map, \(g\) the point-spread function and \(B\) the background light. The transmission map decays exponentially with the distance between the captured object and the camera, in a rate that is dependent on the light wavelength.

For an RGB image, the UIFM can then be expressed as follows, for each color channel \(c\in\{r, g, b\}\) and pixel \(x\),
\begin{equation}\label{eq:uifm}
  I^c(x) = J^ct^c(x) + \left(g^c\ast\left(J^c t^c\right)\right) (x) +  B^c\left(1 - t^c(x)\right).
\end{equation}

Each addend in the above formula corresponds to one component of the original UIFM, in order of appearance both in the equation and the text.

In this work, a further modification of the model is made. By noting that summing the term \(J^ct^c\) is equivalent to summing \(\delta\ast(J^ct^c)\) (where \(\delta\) denotes a Dirac delta), and using the linearity of the convolution, it is possible to combine the first two addends into one without
losing generality. Moreover, a noise component is summed, which is common in the literature \cite{xie2021variational}. The resulting equation is as follows:
\begin{equation}\label{eq:physical-model}
  I^c(x) = B^c\left(1 - t^c(x)\right) + \left(g^c\ast\left(J^c t^c\right)\right) (x) + \xi^c(x),
\end{equation}
where \(\xi(x)\) is random noise on the pixel \(x\), which will be modelled as white noise (i.i.d. centered gaussians). The function \(g\) is no longer a point-spread function but a more abstract convolution kernel instead.

From an image processing standpoint, the desired image \(J\) is ideally obtained by inverting one of the formulae above from the captured image \(I\). However, due to the ill-posedness and analytical untractability of the deconvolution problem involved, it is convenient to simplify the formula by assuming \(g^c\) to be a Dirac delta for each channel \(c\). The resulting equation is
\begin{equation}\label{eq:simplified-physical-model}
  I^c(x) = B^c\left(1 - t^c(x)\right) + \left(J^c t^c\right) (x) + \xi^c(x),
\end{equation}
This is equivalent to ignoring the forward-scattering component and a common practice in the literature \cite{ancuti2017color}. It is often called ``simplified UIFM'', and can still produce good results, but is nevertheless a biased model which completely neglects forward scattering.

\subsection{Formalization of the problem}\label{subsec:formalization}
Our approach consists of firstly estimating \(B\), \(t\) and a preliminary version of \(J\) with traditional methods by assuming the simplified equation \cref{eq:simplified-physical-model} holds. Then, we utilize this preliminary version as initialization of an unfolded variational approach to invert the full \cref{eq:physical-model}.

More concretely, since \(t\) is fixed, we perform a substitution \(Jt\to J\) (which is to be inverted later) and pose the following minimization problem, separately for each channel \(c\in\{r, g, b\}\):
\begin{equation}\label{eq:full-I2-functional}
  \argmin_{(J, g)\in D} E(J, g) = \frac12\|J\ast g - R^c\|_2^2 + \mathcal{R}_1(J) + \mathcal{R}_2(g),
\end{equation}
where \(D = L^2(\Omega)\times L^2(\Omega)\), \(R = I - B(1-t)\) and \(\mathcal{R}_1\) and \(\mathcal{R}_2\) are some continuous regularization functions from \(L^2(\Omega)\) to \(\mathbb{R}\).

This problem is still mathematically intractable, since the resulting functional is not convex due to the convolution being bilinear w.r.t. the variable functions. A further simplification is made, whereby the functional is alternatively minimized by each of the two variables separately instead of jointly. To compensate for this simplification, the process is repeated a few times:
\begin{equation}\label{eq:greedy-iterations}
  \left\{\begin{split}
    g_{m+1}^c &= \argmin_{g\in L^2(\Omega)} E(J_m^c, g) \\
    J_{m+1}^c &= \argmin_{J\in L^2(\Omega)} E(J, g_{m+1}^c)
  \end{split}\right.
\end{equation}
Henceforth, the iterations resulting from this simplification will be indexed with the letter \(m\) and referred to as \textbf{greedy iterations}.

Due to the symmetry of the target functional \(E\), both subproblems can be studied under the same scheme. To see this, consider the following minimization problem: let \(s, y\in L^2(\Omega)\) and \(R\colon L^2(\Omega)\to\overline{\mathbb{R}}\). Fixed those symbols, the idea is to solve for

\[
  \argmin_{x\in L^2(\Omega)}F(x) = \frac12\|x\ast y - s\|_2^2 + R(x)
.\]

Both problems in \cref{eq:greedy-iterations} can be trivially reduced to one of this form by appropriately setting \(s\) to \(I_2\), and \(y\) and \(R\) to either \(J_m\) and \(\mathcal{R}_2\) or \(g_{m+1}\) and \(\mathcal{R}_1\). Thus, studying this problem will yield solutions for both subproblems above.

First, it is necessary to show that it is well-posed, meaning that the operations taking place are well-defined and a solution exists. Define the auxiliary linear operator \(K\colon L^2(\Omega)\to L^2(\Omega)\) given by

\[
  Kx=x\ast y.
.\]

To see that the codomain of this function is, in fact, \(L^2\), we employ the Young inequality together with the well-known fact that \(L^2(\Omega)\subseteq L^1(\Omega)\) for a finitely-measured \(\Omega\):

\[
  \|x\ast y\|_2 \leq \|x\|_{2}\|y\|_{1} < +\infty
.\]

This also implies the boundedness of \(K\).

Also define \(N\colon L^2(\Omega)\to \mathbb{R}\) by \(N(a) = \|a - s\|_{2}^2\), which is clearly strictly convex. In this way, we can express

\[
  F(x) = N(Kx) + R(x),
\]

which is a problem of the same shape of that of \cref{thm:chambolle-pock}. Note that \(N\) is coercive, convex, lower semicontiuous and, in this case, always finite-valued. Convexity and lower semicontinuity imply weak lower semicontinuity.

Therefore, for a proper and weakly lower semicontinuous regularization \(R\) that is either coercive or bounded below,
\(F=N\circ K + R\) is coercive, proper and weakly lower semicontinuous by \cref{lemma:sum-of-lsc,lemma:precomposition-with-linear}, and the problem admits a solution by \cref{thm:direct-method}. For a non-null value of \(y\), the function \(K\) is injective (cf. \cref{prop:injectivity-of-convolution} in the appendix), and therefore the strict convexity of \(N\) implies that of \(N\circ K\), and if \(R\) is convex, then \(F\) is strictly convex. Therefore, the solution is unique. Moreover, the hypotheses of the Chambolle-Pock algorithm are satisfied. It yields the following iterates:

\begin{equation}\label{eq:iterates}
  \left\{
  \begin{split}
    x_{n+1} & = \text{prox}_{\tau R}(x_n - \tau K^*z_n) \\
    \tilde{x}_{n+1} & = 2x_{n+1} - x_n \\
    z_{n+1} & = \text{prox}_{\sigma N^*}(z_n + \sigma K\tilde{x}_{n+1})
  \end{split}
  \right.
\end{equation}
Henceforth, the iterations in a primal-dual scheme will be indexed with the letter \(n\) and referred to as \textbf{stages}.

Closed-form expressions for \(K^*\) and \(\text{prox}_{\sigma N^*}\) can be derived analytically (cf. \cref{prop:proximity-square-norm,prop:proximity-convolution} in the appendix). They are given by

\begin{equation}\label{eq:primal-dual-analytical}
  \text{prox}_{\sigma N^*}(x) = \frac{x-\sigma s}{\sigma + 1} ~~ \text{and} ~~ K^*z = z\ast \overline{y},
\end{equation}
where \(\overline{y}(x) = y(-x)\).

The only ``missing'' symbol is \(\text{prox}_{\tau R}\), but this depends entirely on the regularization function used and has no general closed form.

We now discuss precisely which regularization is to be used for each subproblem. Going forward, the variable \(g\) will be called \textit{kernel} and the variable \(J\) will be called \textit{image}\footnote{Here, \(J\) does not correspond to the radiance (i.e. ``recovered \textit{image}''), but \(J/t\) does. This is just convenient notation.}.

\subsubsection{Kernel regularization}
For the kernel \(g\), it is desirable to impose three constraints:
\begin{enumerate}
  \item \label{kernel-regularization-1}That it is a convolution kernel, i.e., nonnegative and \(\int_{\Omega}g~d\mu = 1\). The assumption that the intensity of \(g\) be \(1\) is motivated by the empirical fact that the simplified UIFM, which is equivalent to setting \(g=\delta\), often yields acceptable results. Convolving by a kernel with a different total intensity (in a practical sense, \(\int_{\Omega}\delta~d\mu=1\)) would alter the intensity of the output image \(J\). This was the case in early experiments where this condition was not imposed.
  \item That it vanishes outside a square much smaller than \(\Omega\). This condition is ubiquitous for convolution kernels in computer vision, as it corresponds to their discrete representations having much smaller sizes than the images'.
  \item That it is a decreasing function of its radius. That is, that there exist a decreasing function \(\varphi\colon [0, +\infty)\to \mathbb{R}\) such that \(g(x)=\varphi(\|x\|)\) for all \(x\in\Omega\). This assumption is motivated by two intuitions: first, that there are no privileged directions in an underwater setting, i.e. \(g\) is rotationally invariant, or, equivalently, a function of its radius; second, that the larger the angle of the scattering, the larger the distance that light must travel from the object to the camera, and therefore the more intensity
  is absorbed from it.
\end{enumerate}

These constraints can all be expressed via a regularization function as follows: let \[\mathcal{K}=\{f\in L^2(\Omega) | f\geq 0, \|f\|_{1}=1\}.\] Then, condition 1 is equivalent to \(g\in \mathcal{K}\). Additionally, for a fixed \(r\in\mathbb{Z}^+\) such that \([-r, r]^2\subseteq\Omega\), define \[\mathcal{S}_r = \{f\in L^2(\Omega) | f\equiv 0 \text{~in~} \Omega\setminus [-r, r]^2\}.\] Condition 2 is equivalent to \(g\in \mathcal{S}_r\) for some sufficiently small \(r\). Finally, define \[\mathcal{D}=\{\left.\varphi\circ\|\cdot\|_{2} ~\right|~ \varphi\colon[0,+\infty)\to \mathbb{R} \text{ is decreasing}\},\] so that condition 3 is equivalent to \(g\in \mathcal{D}\).

It is trivial to check that the sets \(\mathcal{K}\), \(\mathcal{S}_r\) and \(\mathcal{D}\) are convex. It is also possible to see that they are closed in \(L^2(\Omega)\) (cf. \cref{prop:closed-sets}), and hence their intersection is closed and convex. Thus, the indicator function given by

\[\delta_{\mathcal{K}\cap \mathcal{S}_r\cap \mathcal{D}}(g) = \left\{\begin{array}{rl}
  +\infty ,& \text{ if }g\in\mathcal{K}\cap \mathcal{S}_r\cap \mathcal{D},\\
  0       ,& \text{ otherwise,}
\end{array}\right.\]

is convex and weakly lower semicontinuous by Lemma 2.5.ii) of \cite{clason2024introductionnonsmoothanalysisoptimization}. It is bounded below, and thus, taking \(R=\delta_{\mathcal{K}\cap \mathcal{S}_r\cap \mathcal{D}}\) would guarantee existence of solution to the problem.

However, computing the proximity operator resulting would be challenging, so a final simplification is made. Specifically, \(g\) is taken as a centered gaussian density truncated to \([-r,r]^2\) (that is, set to 0 outside of \([-r, r]^2\)) and normalized so that its integral is \(1\). This is a sufficient condition for those outlined above, and allows for an important simplification of the optimization scheme: instead of performing primal-dual iterations, a simple gradient descent will be used to optimize the standard deviation of the gaussian density.

Such simplification may not reach the optimum guaranteed to exist with the conditions above, but will be empirically sufficient, likely faster to converge and much simpler from an implementation standpoint.
\subsubsection{Image regularization} Instead of fixing a regularization for the image, the iterative scheme \cref{eq:iterates} will be unfolded.

Concretely, the proximity operator \(\text{prox}_{\tau R}\) will be replaced by neural networks that will be then trained end-to-end. Mathematically, the resulting scheme is
\begin{equation}\label{eq:unfolded-iterates}
  \left\{
  \begin{split}
    x_{n+1}^m & = \mathcal{N}_n^m(x_n^m - \tau K^*z_n^m) \\
    \tilde{x}_{n+1}^m & = 2x_{n+1}^m - x_n^m \\
    z_{n+1}^m & = \text{prox}_{\sigma N^*}(z_n^m + \sigma K\tilde{x}_{n+1}^m)
  \end{split}
  \right,
\end{equation}
where \(\mathcal{N}_n^m\) is a neural network specific for stage \(n\) in the greedy iteration \(m\). Although this deviates more from the original primal-dual scheme than using a fixed neural network would, it has been shown to yield significantly better results \cite{8271999}.

Instead of iterating until convergence, the number of stages and greedy iterations will be fixed beforehand.

\section{Algorithmic approach}

From a high-level perspective, the approach followed to obtain an estimate of the radiance \(J\) from the image \(I\) is as follows:

\begin{enumerate}
  \item Use a score-based quad-tree algorithm to estimate the background light \(B\).
  \item Assume the simplified model, \cref{eq:simplified-physical-model}, holds and use a DPC prior to obtain a coarse estimate of the red channel of the transmission map \(t^r\). Refine this estimate by passing it through a guided filter with \(I\) as a guide.
  \item Obtain the green and blue channels of the transmission map by exponentiating the red one. Then, use the obtained estimate of \(t\) to compute an initial estimate for \(J\), \(J_0\).
  \item Pose a minimization problem with the obtained estimates for \(B\) and \(t\) and the full model, \cref{eq:physical-model} with a formal substitution \(Jt\to J\). Solve this problem with an alternating iterative approach using \(J_0t\) as initialization, and obtain a final estimate for \(J\) from the minimum \(J^*\) by inverting the substitution: \(J=J^*/\max(t, 0.1)\). The kernel is optimized with simple gradient updates, and the radiance is optimized via \cref{eq:unfolded-iterates}.
\end{enumerate}
In the following, steps 1 to 3 will be referred to as \textit{deterministic steps} and step 4 will be called \textit{variational step}.

\subsection{Determinisitc steps}
For step 1, the background light is estimated by iteratively splitting the image into four quadrants. Out of those regions, the one with the highest score, given by the average minus the standard deviation of the pixel values in the region, is selected and split again. The process stops once the selected region is smaller than a given threshold (\(16\) colored pixels in this implementation). Then, the background light of the image is chosen as the pixel closest to white (in euclidean norm) in that small patch. This process is summarized in Algorithm \ref{al:background-light}.

For step 2, a Red Channel Prior is utilized \cite{GALDRAN2015132}. This is a
modification of the classical Dark Channel Prior \cite{5567108}. The
approach is based on the assumption that, for every pixel \(x\in\Omega\), the following equality holds for a square patch of fixed radius centered at \(x\), \(\Omega_x\):
\[
  \min_{y\in\Omega_x}\min(1-I^r(y), I^g(y), I^b(y), S(y)) = 0
,\]
where the \textit{saturation map} \(S\) is defined as \(0\) for black pixels and \(S=1-\frac{\min(I^r, I^g, I^b)}{\max(I^r, I^g, I^b)}\) otherwise.
Additionally, \(t_0\) is imposed to be constant across channels and in \(\Omega_x\) neighborhoods. From this and \cref{eq:simplified-physical-model}, authors in \cite{GALDRAN2015132} derive the following equation for the transmission map of the red channel, \(t_0^r\), for a fixed multiplier \(\lambda\in[0,1]\), which in this work is set to \(0.75\):
\[
  t_0^r(x) = 1 - \min_{y\in\Omega_x}\left(\min\left(\frac{1-I^r(y)}{1-B^r}, \frac{I^g(y)}{B^g}, \frac{I^b(y)}{B^b}, \lambda S(y)\right)\right)
.\]
\Frank{Nota: he estat incapaç de reproduir aquest raonament amb la saturació. És a dir, de s'RCP ``estàndard'' (llevant es \(S(y)\) des mínim) sí que se dedueix la fórmula (també sense \(S(y)\)) amb les hipòtesis que explic, però no he estat capaç d'introduir-li la saturació. A l'article no donen massa pistes llevat que és una adaptació ``straight-forward'' de s'altre raonament. Si vos sembla bé, jo ho deixaria així com està escrit.}

For step 3, a fine version of the transmission map, \(t^r\), is obtained by passing \(t_0^r\) through a guided filter \cite{6319316} with \(I\) as the guide. The transmission map estimates for the other two channels are obtained by
exponentiating the red transmission map by appropriate coefficients. This is sound since, physically, \(t\) decays exponentially with the distance from the object to the camera, and the base of the exponent depends on the color channel. Thus, the transmission maps for the different color channels are related via exponentiation. In this work, the exact coefficients presented in \cite{7574330} are utilized:
\[
  t^c = (t^r)^\frac{\alpha_c}{\alpha_r}
,\]
where \(c\in\{r,g,b\}\) and \(\alpha_c\) depends on the color's wavelength \(\lambda_c\) in nanometers (respectively: \(620, 540\) and \(450\) for red, green and blue) and the background light \(B^c\):
\[
  \alpha_c = \frac{-0.00113\lambda_c + 1.62517}{B^c}
.\]

Finally, we estimate a first version of \(J_0t\) by inverting \cref{eq:simplified-physical-model}. For each color channel \(c\in\{r, g, b\}\):
\begin{equation}\label{eq:J0}
  (J^c_0t^c)(x) = I^c(x) - (1-t^c(x))B^c
\end{equation}
Of course, it would be possible to compute \(J_0\) directly from the previous equation, but since it is numerically unstable (\(t\) can be and often is close to \(0\)) and a formal substitution \(Jt\to J\) will be performed for the subsequent variational problem anyway, this is unnecessary.

Steps 2 and 3 are summarized in Algorithm \ref{al:I1}.

\begin{algorithm}\label{al:background-light}
\caption{Estimate background light.}
\KwData{\(I\)}
\KwResult{\(J\)}

img \(\gets I\)\;
\While{Size(img) > min\_size}{
  quadrants \(\gets\) Split(img)\;
  max\_score \(\gets 0\)\;
  \For{\(q\) in quadrants}{
    \(\mu\gets\) Mean(\(q\))\;
    \(\sigma\gets\) Std(\(q\))\;
    score \(\gets \mu - \sigma\)\;
    \If{score > max\_score}{
      max\_score \(\gets\) score\;
      img \(\gets q\) \;
    }
  }
}
\end{algorithm}
\begin{algorithm}\label{al:I1}
\caption{Estimate TM and initial radiance.}
\KwData{Input image \(I\), background light \(B\), TM patch radius \(r_t\), guided filter patch radius \(r_g\)}
\KwResult{TM estimation \(t\), multiplied radiance estimation \(J_0t\)}

\Comment*[l]{Compute coarse TM}
\(S\gets \) Saturation(\(I\))\;
Initialize \(t^r\) with the width and height of \(I\)\;
\For{pixel \(x\)}{
  \Comment*[l]{\(\Omega_x\) patch around \(x\), radius \(r_t\)}
  r\_min \(\gets \min_{y\in \Omega_x}\)(\(1 - I^r\)[y])\(/(1-B^r)\)\;
  g\_min \(\gets \min_{y\in \Omega_x}\)(\(I^g\)[y])\(/B^g\)\;
  b\_min \(\gets \min_{y\in \Omega_x}\)(\(I^b\)[y])\(/B^b\)\;
  s\_min \(\gets \min_{y\in \Omega_x}\)(\(S\)[y])\(\cdot \lambda\)\;

  \(t^r[x] \gets 1 -\min\)(r\_min, g\_min, b\_min, s\_min)
}
\Comment*[l]{Refine transmission map}
\(t^r \gets GuidedFilter(I, t^r; r_g)\)\;
Obtain \(t\) by exponentiating \(t^r\)\;

\Comment*[l]{Estimate \(J_0t\)}
\For{channel \(c\)}{
  \((J_0^ct^c)\gets I^c - (1-t^c)B^c\)
}
\Return{\(t, J_0t\)}
\end{algorithm}
\subsection{Variational step}
For step \(4\), the minimization subproblems in \cref{eq:greedy-iterations} will be solved for a fixed, small number of greedy iterations (concretely, 3). The iterative scheme in \cref{eq:iterates} with the formulae in \cref{eq:primal-dual-analytical} is used, with the aforementioned symbolic substitutions. The missing operator \(\text{prox}_{\tau R}\) is unfolded, meaning that it is substituted by a neural network instead of explicitly derived. The architecture and training of said neural network will be discussed later, in the implementation section.

The final approach used for obtaining an approximate solution to \cref{eq:full-I2-functional} is summarized in Algorithm \ref{al:I2-estimation}. Instead of analytically computing the derivative of \(\|I-J\ast g\|_2^2\) w.r.t. the standard deviation of \(g\), \textit{pytorch}'s \textit{autograd} engine is leveraged. An ADAM optimizer is used with a step size of \(\sigma\).

\begin{algorithm}\caption{Solve variational problem.}
\label{al:I2-estimation}
  \KwData{\(I, B, t, J_0t, \sigma, \tau\)}
\KwResult{\(J\)}
\(J\gets J_0t\)\;
\(R\gets I - (1-t)B\)\;
Initialize \(g\) as a centered gaussian kernel with std. of 1\;
Initialize dual variables \(\tilde{g}\), \(\tilde{J}\) to 0\;
\For{\(n=1\) \KwTo \(G\)}{
  \Comment*[l]{Fix \(J\), estimate \(g\).}
  For a fixed, small number of iterations, minimize
  \(\|R-J\ast g\|_2^2\) via gradient updates on the std.

  \Comment*[l]{Fix \(g\), estimate \(J\).}
  \(\overline{g} \gets DoubleFlip(g)\)\;
  \For{\(s=1\) \KwTo \(S\)}{
    \(tmp \gets NeuralNet_{ns}^J\left(J - \tau\cdot\left(\tilde{J} * \overline{g}\right)\right)\)\;
    \(J\gets 2 \cdot tmp - J\)\;
    \(\tilde{J}\gets (\tilde{J} + \sigma\cdot (J* g) - \sigma \cdot R)/(\sigma + 1)\)\;
  }
}
\Return J / t
\end{algorithm}

\subsection{Implementation}
\Frank{Xerrar de U2Fold, la implementació en pytorch dels algoritmes d'abans i de la UNET additiva, etc.}

The algorithms detailed above were implemented in the Python programming
language, version 3.12. The major dependency used was \href{https://pytorch.org/}{\textit{pytorch}}, which
provides a nice programming interface for manipulating multidimensional arrays 
via the \textit{Tensor} abstraction, automatically performing simple
optimization algorithms via its \textit{autograd} engine, and, what it is most
popular for, implementing neural networks with the \textit{torch.nn.Module}
class and a variety of related utilities. A secondary dependency is
\href{https://pydantic.com.cn/en/}{\textit{pydantic}}, for validating input parameters specified in a JSON file
such as network hyperparameters or the dataset path.

The source code for the program is freely available at \href{https://github.com/Frankwii/u2fold}{https://github.com/Frankwii/u2fold}.

\section{Experimentation}
\subsection{Algorithmic selection of hyperparameters}
Given the high computational costs of training neural networks, it is unfeasible to perform a grid search on a substantial set of hyperparameter choices and train a network from scratch for each choice.

As an attempt to overcome this limitation, the following approach is proposed. For a given neural network architecture:
\begin{enumerate}
  \item Determine, through a manual process, a sensible hyperparameter combination as a starting point. This includes the choice of \textbf{training-related} hyperparameters, such as the optimizer, learning rate scheduler, etc., and \textbf{architectural} hyperparameters, such as the number of layers, their sizes, activation functions...
  \item Fix the training-related hyperparameters of the starting point, and for a given set of choices for the architectural hyperparemters (which should include the starting point), train a model on each choice for a small number of epochs. Select the ``best'' one by minimizing a target metric.
  \item Fix the architectural parameters of the best neural network so far, and, similarly, select the best choice of training-related hyperparameters by training a model on each choice. This time, however, the number of epochs should be larger, since some learning rate schedulers show performances greatly dependent on the number of epochs.
  \item Finally, train the best model so far during many epochs as an attempt to get the best performance.
\end{enumerate}
\Frank{Nota: això ha anat bastant malament. Crec que hauré de tornar a sa manera manual de simplement anar provant combinacions i au. Una llàstima.}
Of course, the target metric choice plays a central role in determining the output. Several metrics will be used to evaluate the quality of the model (cf. [TODO: AÑADIR SECCIÓN DE MÉTRICAS]), but it would be desirable to obtain a single number that will allow the automatic comparison of models. An obvious choice is a simple addition of the metrics, but this presents the problem of different metrics having different ranges of values. A simple solution to this
is dividing each value by its average computed over the dataset.

\begin{table}[h]
\centering
\caption{Calibration results for different metrics and loss terms with the UIEB dataset.}
\begin{tabular}{|c|c|c|}
\hline
\textbf{Metric or loss} & \textbf{Average} & \textbf{Standard deviation} \\ \hline
UCIQEm                  & 10.59041         & 2.16975                     \\ \hline
TV                      & 0.05879          & 0.03238                     \\ \hline
PSNRm                   & 0.05977          & 0.02899                     \\ \hline
DSSIM                   & 0.11068          & 0.06204                     \\ \hline
MSE                     & 0.02561          & 0.02265                     \\ \hline
CCSm                    & 0.03598          & 0.03860                     \\ \hline
Fidelity loss           & 0.03710          & 0.02991                     \\ \hline
\end{tabular}
\end{table}


\section{Conclusions}

\section{Student's comments}
This section is not strictly part of the work but instead it is intended to offer some insight on its writing to reviewers of the thesis or potential future students that use it as a reference for similar works.
\subsection{Limitations and ideas for future work}
\Frank{
Ideas:
\begin{enumerate}
  \item El algoritmo actual está limitado por el hecho de que la manera de estimar B, t, etc. es relativamente rudimentaria y, más importante, fija. Si está mal estimada la t al principio va mal todo lo demás.
  \item Probar de utilizar una red neuronal para seleccionar hiperparámetros adecuados a cada imagen: exponentes para transmission map, coeficientes de saturación o regularización, etc.
  \item Probar con otros tipos de redes neuronales (ViT, ResNet...)
  \item Probar con maneras más sofisticadas de estimar la \(g\) (p.ej. no suponer que es gaussiano, sino de una familia más general).
  \item Probar a hacer unfolding no solo con uno de los proximales sino con los dos.
  \item Probar de compartir los pesos de la red (aunque esto probablemente vaya peor)
\end{enumerate}
}
\subsection{Use of AI in this work}
Large Language Models (LLMs) were used mostly as a consult material helpful to build early intuition. Concretely, the Gemini 2.5 Pro model available for free at \href{https://aistudio.google.com}{https://aistudio.google.com} (at the moment of writing this) was used. Personally, I find the base model useful for scientific tasks and the lack of memory across chats by default to be an advantage instead of otherwise, since it forced me to write specific prompts for the concrete task I was interested in early on, instead of having a long, ongoing conversation. That often led me to realizing the solution to the problem myself while articulating the question.

Almost every line of code under the ``src/u2fold'' directory of the provided repository was written by me and not automatically generated. I generally strongly dislike AI-generated code since it tends to contain very verbose comments, many unnecessary checks or early returns and generic names that do not reflect the intention behind the code. I subscribe to the philosophy that inline comments (not docstrings) tend to get outdated extremely quickly on relatively large codebases such as this thesis', and that the programmer should name their abstractions and structure their code in such a way that, generally speaking, it is understandable to someone external but fluent in the language just by reading it, with no need for intercalated explanations.

That being said, I did use AI tools for the following coding-related scenarios:
\begin{itemize}
  \item Write parts of code snippets under the ``scripts'' directory since those were intended to be ran few times and not be modified afterwards, and were small enough for manual inspection and polishing. The open-source tool \href{https://github.com/google-gemini/gemini-cli}{Gemini CLI} proved especially useful for this.
  \item ``Code reviews'' of certain modules. Although very good for catching obvious mistakes early, this proved to be of limited usefulness for larger or more important modules since current LLMs often show sycophantic behavior and will agree on retrospectively dubious architectural choices. This led to a false sense of security that was sometimes counterproductive.
\end{itemize}

For the mathematical part of the thesis, I found the utilization of LLMs
incredibly useful not because of the reasonings they present, which are often
irrecoverably flawed in very subtle and hard to spot ways (hence, a waste of
time to pay attention to), but because they do
identify useful sources of information or relatively unknown but certainly related theorems. I would often ask for a proof of some conjecture I had come up with and then completely ignore the answer except for the references it provided. I would then go through the references and find something that was indeed useful. An example of this was the Titchmarsh-Lions theorem, which I was unaware of but was crucial to prove the uniqueness of solution for the greedy subproblems.

Finally, they were used for styling and spelling review.

\subsection{Difficulties found while writing the thesis}
\Frank{
Comentar lo poco concreto que era el artículo de referencia y lo complicado que fue de seguir. Concretamente, la manera en la que estiman el transmission map es 0 explícita y especialmente no se entiende qué hacen para estimar la \(g\).
}

One particular limitation to be difficulty I encountered was the pitiful and ascientific, yet ubiquitous practice of not publishing
source code along with research papers, but only a compiled binary instead. This is done
possibly in the hopes of being cited when compared to other approaches, but discouraging researches from reproducing or improving their methods. This hinders
one of the pillars of science, reproducibility, and should not be allowed in
a journal that considers itself serious and scientific. This was the case for \cite{7574330} and, most importantly, the main reference \cite{xie2021variational}, which made fine-tuning algorithmic parameters or
filling in details of the (often poor) exposition much harder than needed.

\Frank{También tal vez la mala idea que fue la aplicación de CLI (funciona mucho mejor configurarlo con un JSON)}
\subsection{Original contributions}
Paper \cite{xie2021variational} served as an initial reference for this work. However, several modifications were made, many of which were fully original.

The mathematical model used is, of course, a classical one in the field and the same as in \cite{xie2021variational}. Its formalization and treatment, however, are original. Concretely, to the best of the author's knowledge, the ``unification'' of forward-scattering and direct components into a single one has
not been done previously in the literature. All of the mathematical treatment
and justifications (existence and uniqueness of solutions to the problems...) of
the derived variational approach are also novel.

The additive UNet architecture is new too.

The implementations of all algorithms and neural networks described in this
paper were made from scratch on \textit{pytorch}. In
addition to those, so were the guided filter's and all defined metrics', some of which required colorspace conversions. Free and open
source library \href{https://github.com/kornia/kornia}{kornia's} implementation
of colorspace conversions was consulted, specifically for the coefficients of
the conversion matrices.

\section{Appendix}

\begin{proposition}\label{prop:closed-sets}
  The sets \(\mathcal{K}\), \(\mathcal{S}_r\) and \(\mathcal{D}\) are all closed.
\end{proposition}
\textit{Proof.}
Let \(S\) be one of the sets above, and let \(\{f_n\}_n\subseteq S\) denote a sequence converging in \(L^2(\Omega)\) to a given limit \(f\in L^2(\Omega)\). Showing that the set \(S\) is closed is equivalent to showing that \(f\in S\).

A well-known result about \(L^p\) spaces states that \(L^p\) convergence implies pointwise convergence up to a subsequence \cite{ash1972real}. This result will be used in all three cases. For notational simplicity, simply take \(f_n\) as the converging subsequence.

First, consider \(S=\mathcal{K}\). The result above immediately implies that \(f\) is nonnegative by taking limits on the inequality \(f_n(x) \geq 0\). Another well-known result about \(L^p(\Omega)\) spaces states that, whenever \(\mu(\Omega)<+\infty\), the following inequality holds for \(1\leq p < q < +\infty\) and any measurable function \(g\):
\[
  \|g\|_p \leq \mu(\Omega)^{1/p - 1/q} \|g\|_q
.\]
Thus, for \(p=1\), \(q=2\),
\[
  \|f_n-f\|_1 \leq \mu(\Omega)^{1/2}\|f_n - f\|_2
.\]
Taking limits, this implies that \(f_n\) converges to \(f\) in \(L^1\). As a consequence, \(\|f\|_{1}=\lim_n \|f_n\|_{1}=\lim_n 1 = 1\).

Second, for \(S=\mathcal{S}_r\), the result is immediate by taking limits on the converging subsequence outside of \([-r, r]^2\).

Finally, for \(S=\mathcal{D}\), consider, for each \(f_n\), the function \(\phi_n\) such that \(f_n=\phi_n \circ \|\cdot\|_{2}\). Take some \(x\in L^2(\Omega)\) such that \(\|x\|_{2}=1\). Then, for each \(t\in[0, +\infty)\),
\[
  \phi_n(t) = f_n(tx)
.\]
This defines a function \(\phi\) by \(\phi(t)=\lim_n\phi_n(t)=\lim_nf_n(tx)=f(tx)\). Since \(x\) was arbitrary, it also shows that \(f=\phi\circ\|\cdot\|_{2}\) as desired. \qed

\begin{proposition}[Injectivity of convolution]\label{prop:injectivity-of-convolution}
  Suppose that \(g\) is the continuous representation of a non-null digital image. Then, the function \(K\colon L^2(\Omega)\to L^2(\Omega)\) given by \(Kx = x\ast g\) is injective.
\end{proposition}
\textit{Proof. } Since the mapping has been shown to be well-defined and linear, it suffices to show that \(ker(K)=\{0\}\); that is, that \(Kx=0\) a.e. implies \(x=0\) a.e.

This proof will make use of the Titchmarsh-Lions convolution theorem, which states that for any two compactly supported distributions \(T_1, T_2\), the convex hull of \(supp(T_1\ast T_2)\) is the sum of the convex hulls of \(supp(T_1)\) and \(supp(T_2)\) (chapter 45 of \cite{donoghue1969distributions}). Properly defining distributions, convolutions between them or their supports is out of the scope of this work\footnote{See \cite{donoghue1969distributions} for a detailed exposition.}, but it suffices to know that any \(L^p(\Omega)\) function for \(p\geq 1\) and a bounded measurable set \(\Omega\) defines a distribution, and that the definitions given for distributions correspond exacty to those given for \(L^p(\Omega)\) functions, the latter being much simpler.

The support of a function \(f\in L^p(\Omega)\) in this context is defined as the smallest closed subset of \(\mathbb{R}^2\) such that \(f\) is \(0\) almost everywhere outside of it. Formulaically:
\[
  supp(f) = \bigcap_{A \text{ closed}, f=0 \text{ a.e. in } A^c}A
,\]
where \(f\) is extended to be \(0\) outside of \(\Omega\).

It is immediate to see that, for a bounded and measurable set \(\Omega\subseteq \mathbb{R}^2\), an \(L^p(\Omega)\) function has a compact support, and that said is empty if, and only if, the function is null almost everywhere. It is also immediate to see that a subset of \(\mathbb{R}^n\) is empty if, and only if, its convex hull is empty.

Thus, if \(x\) is such that \(x\ast g=0\) a.e., its support is empty, and so is its convex hull. This implies that either the convex hull of \(supp(g)\) or the convex hull of \(supp(x)\) are empty. By hypothesis, \(g\) is not null, and hence it must be that \(x\) is zero almost everywhere. \qed

In the following two propositions, the usual identification between a Hilbert space and its dual via \(x\mapsto(y\mapsto\langle x,y\rangle)\) is implicitly assumed. This is done, in particular, for \(L^2(\Omega)\).

\begin{proposition}\label{prop:proximity-square-norm}
  Let \(N\colon L^2(\Omega)\to \mathbb{R}\) be defined by
  \[
    N(x) = \|x-s\|_{_2^2}
  \]
  for some \(s\in L^2(\Omega)\). Then,
  \[
  \text{prox}_{\sigma N^*}(x) = \frac{x - \sigma s}{\sigma + 1}
  .\]

\end{proposition}
\textit{Proof.}
By the Moreau decomposition theorem,
\begin{equation}\label{eq:proximity-moreau}
  \text{prox}_{\sigma N^*}(x) = x - \sigma\text{prox}_{N/\sigma}\left(\frac x\sigma\right).
\end{equation}

Thus, it is sufficient to compute \(\text{prox}_{\lambda N}\) for an arbitrary \(\lambda > 0\) and then replace \(\lambda = \frac 1\sigma\).

For a fixed \(z\in L^2(\Omega)\), consider the functional \(J_{z, \lambda}(y) = \frac{\|y - z\|_{2}^2}{2\lambda} + \frac12 \|y - s\|_{2}^2\), so that
\[
  \text{prox}_{\lambda N}(z) = \text{argmin}_{y}J_{z, \lambda}(y)
.\]

Clearly, \(J_{z, \lambda}\) is Gâteaux differentiable and
\[
  J_{z, \lambda}(y; h) = \langle h, \frac 1\lambda(y-z)\rangle + \langle h, y-s\rangle
.\]

Whence \(\forall h\colon J_{z, \lambda}(y; h) = 0\) if, and only if, \(y=\frac{z + \lambda s}{\lambda + 1}\). Thus, by the Fermat principle, and taking \(\lambda = \frac 1\sigma\),
\[
  \text{prox}_{N / \sigma}(z) = \text{argmin}_{y}J_{z, 1/\sigma}(y) = \frac{\sigma z+ s}{\sigma + 1}
.\]

Combining this last equation with \cref{eq:proximity-moreau}, we have that
\[
  \text{prox}_{\sigma N^*}(x) = x -\frac{\sigma}{\sigma + 1}(x+s) = \frac{x - \sigma s}{\sigma + 1}
.\qed\]

\begin{proposition}\label{prop:proximity-convolution}
  Let \(K\colon L^2(\Omega)\to L^2(\Omega)\) be defined by
  \[
    Kx = x\ast y
  \]
  for some \(y\in L^2(\Omega)\). Then, the dual operator \(K^*\) is given by \[K^*x=x\ast\overline{y}.\]
\end{proposition}
\textit{Proof. } By definition, \(K^*\) is the only operator satisfying the equality
\[
  \langle Ka, b \rangle = \langle a, K^* b\rangle
\]
for all \(a, b\in L^2(\Omega)\), where \(\langle\cdot,\cdot\rangle\) is the usual scalar product in \(L^2(\Omega)\): \(\langle f, g\rangle = \int_{\Omega}fg~dm\), where \(m\) is the Lebesgue measure restricted to \(\Omega\).

Note that, in order to properly define convolution between functions with domain \(\Omega\subseteq\mathbb{R}^n\), it is first necessary to extend to \(\mathbb{R}^n\) by setting them to \(0\) in \(\Omega^c\).

It is well-known that for any given \(\varphi, \phi, \psi\colon\mathbb{R}^n\to\mathbb{R}\) such that their pairwise convolutions are defined and integrable, the following equality holds:

\begin{equation}\label{eq:convolution/inversion}
  \langle\varphi\ast\phi, \psi\rangle = \langle\phi, \psi\ast\overline{\varphi}\rangle
\end{equation}

where \(\overline{\varphi}(x) = \varphi(-x)\). Therefore,
\[
  \langle a\ast y, b\rangle = \langle a, b\ast\overline{y} \rangle
.\]

Thus, \(K^*\) is simply given by \(K^*b = b\ast \overline{y}\). \qed

\bibliographystyle{abbrv}
\bibliography{bibliography} % EDIT THE FILE `bibliography.bib` WITH YOUR REFERENCES

\end{document}
